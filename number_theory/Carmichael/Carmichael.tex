\section{Carmichael function}
% $\lambda(n)$ 다음을 만족하는 가장 작은 양의 정수 $m$

% $a^m \equiv 1 \pmod{n}$

% $n = p_1^{e_1} \cdots p_r^{e_r}$

% $\lambda(n) = \text{lcm}(\phi(p_1^{e_1}), \ldots,\phi(p_r^{e_r}))$

\url{http://www.gutenberg.org/files/13693/13693-pdf.pdf}
    



Then let us write
\begin{gather}
a^{p-1} = 1 + hp. \tag{1} \\
\intertext{Raising each member of this equation to the
$p^{\text{th}}$ power we may write the result in the form}
a^{p(p-1)} = 1 + h_1p^2. \tag{2} \\
\intertext{where $h_1$ is an integer. Hence}
a^{p(p-1)} \equiv 1 \bmod p^2. \notag \\
\intertext{By raising each member of (2) to the $p^{\text{th}}$
power we can readily show that}
a^{p^2(p-1)} \equiv 1 \bmod p^3. \notag \\
\intertext{It is now easy to see that we shall have in general}
a^{p^{\alpha - 1}(p-1)} \equiv 1 \bmod p^{\alpha}. \notag \\
\intertext{where $\alpha$ is a positive integer; that is,}
a^{\phi(p^{\alpha})} \equiv 1 \bmod p^{\alpha}. \notag
\end{gather}


    
For the special case when $p$ is 2 this result can be extended. For
this case (1) becomes
\begin{gather}
a = 1 + 2h. \notag \\
\intertext{Squaring we have}
a^2 = 1 + 4h(h+1). \notag \\
\intertext{Hence,}
a^2 = 1+8h_1, \tag{3} \\
\intertext{where $h_1$ is an integer. Therefore}
a^2 \equiv 1 \bmod 2^3. \notag \\
\intertext{Squaring (3) we have}
a^{2^2} = 1 + 2^4h_2; \notag \\
\intertext{or}
a^{2^2} \equiv 1 \bmod 2^4. \notag \\
\intertext{It is now easy to see that we shall have in general}
a^{2^{\alpha-2}} \equiv 1 \bmod 2^{\alpha} \notag \\
\intertext{if $\alpha > 2$. That is,}
a^{\frac{1}{2}\phi(2^{\alpha})} \equiv 1 \bmod 2^{\alpha}
  \text{ if } a > 2.
\end{gather}

Now in terms of the $\phi$-function let us define a new function
$\lambda(m)$ as follows:
\begin{align*}
\lambda(2^{\alpha}) &= \phi(2^{\alpha}) \text{ if $a = 0, 1, 2$;} \\
\lambda(2^{\alpha}) &= \frac{1}{2}\phi(2^{\alpha})
                                               \text{ if $a > 2$;} \\
\lambda(p^{\alpha}) &= \phi(p^{\alpha})
                                   \text{ if $p$ is an odd prime;} \\
\lambda(2^{\alpha} p_1^{\alpha_1} p_2^{\alpha_2} \cdots p_n^{\alpha_n}) 
&= \text{lcm}(
    \lambda(2^{\alpha}),
    \lambda(p_1^{\alpha_1}),
    \lambda(p_2^{\alpha_2}), \ldots, \lambda(p_n^{\alpha_n}
    )
)
\end{align*}

$2, p_1, p_2, \ldots, p_n$ being different primes.%
\index{$\lambda(m)$}

Denote by $m$ the number
\begin{equation*}
m = 2^{\alpha}p_1^{\alpha_1}p_2^{\alpha_2} \cdots p_n^{\alpha_n}.
\end{equation*}
Let $a$ be any number prime to $m$. From our preceding results we
have
\begin{align*}
a^{\lambda(2^{\alpha})}     &\equiv 1 \bmod 2^{\alpha}, \\
a^{\lambda(p_1^{\alpha_1})} &\equiv 1 \bmod p_1^{\alpha_1},\\
a^{\lambda(p_2^{\alpha_2})} &\equiv 1 \bmod p_2^{\alpha_2}, \\
\ldots \\
a^{\lambda(p_n^{\alpha_n})} &\equiv 1 \bmod p_2^{\alpha_n}. \\
\end{align*}

Now any one of these congruences remains true if both of its members
are raised to the same positive integral power, whatever that power
may be. Then let us raise both members of the first congruence to
the power $\frac{\lambda(m)}{\lambda(2^\alpha)}$; both members of
the second congruence to the power
$\frac{\lambda(m)}{\lambda(p_1^{\alpha_1})}$; $\ldots$; both members
of the last congruence to the power
$\frac{\lambda(m)}{\lambda(p_n^{\alpha_n})}$. Then we have
\begin{align*}
a^{\lambda(m)} &\equiv 1 \mod 2^\alpha, \\
a^{\lambda(m)} &\equiv 1 \mod p_1^{\alpha_1}, \\
\ldots \ldots \\
a^{\lambda(m)} &\equiv 1 \mod p_n^{\alpha_n}. \\
\intertext{From these congruences we have immediately}
a^{\lambda(m)} &\equiv 1 \mod m.
\end{align*}

We may state this result in full in the following theorem:

\smallskip \emph{If $a$ and $m$ are any two relatively prime positive
integers, the congruence}
\begin{equation*}
a^{\lambda(m)} \equiv 1 \mod m.
\end{equation*}
\emph{is satisfied.}

As an excellent example to show the possible difference between the
exponent $\lambda(m)$ in this theorem and the exponent $\phi(m)$ in
Fermat's general theorem, let us take
\begin{gather*}
m = 2^6 \cdot 3^3 \cdot 5 \cdot 7 \cdot 13 \cdot 17 \cdot 19
        \cdot 37 \cdot 73. \\
\intertext{Here}
\lambda(m) = 2^4 \cdot 3^2, \quad \phi(m) = 2^{31} \cdot 3^{10}.
\end{gather*}

In a later chapter we shall show that there is no exponent $\nu$
less than $\lambda(m)$ for which the congruence
\begin{equation*}
a^\nu = 1 \mod m
\end{equation*}
is verified for every integer $a$ prime to $m$.

From our theorem, as stated above, Fermat's general theorem follows
as a corollary, since $\lambda(m)$ is obviously a factor of
$\phi(m)$,
\begin{equation*}
\phi(m) = \phi(2^\alpha) \phi(p_1^{\alpha_1}) \ldots
               \phi(p_n^{\alpha_n}).
\end{equation*}

%%%%%%%%%%%%%%%%%%%%%%%%%%%%%%%%%%%%%%%%%%%%%%%%%%%%%%%%%%%%%%%%%%%%%%%%%%%%%%%%%%%%%%%%%%%%%%%%%%%%%%%%%%%%%%%%%%%%%%%%%%%%%%%%%%%%%%%%%%%%%%%%%%%%%%%%%%%%%%%%%%%%%%%%

\subsection{Solve 31.8-2(CLRS)}
\begin{justbox}

    It is possible to strengthen Euler's theorem slightly to the form

    $a^{\lambda(n)} \equiv 1 (\mod n)$ for all $a \in \mathbb Z_n^*$,
    
    where $n = p_1^{e_1} \cdots p_r^{e_r}$ and $\lambda(n)$ is defined by
    
    $$\lambda(n) = \text{lcm}(\phi(p_1^{e_1}), \ldots, \phi\phi(p_r^{e_r})). $$
    
    Prove that $\lambda(n) \mid \phi(n)$. A composite number $n$ is a Carmichael number if $\lambda(n) \mid n - 1$. The smallest Carmichael number is $561 = 3 \cdot 11 \cdot 17$; here, $\lambda(n) = \text{lcm}(2, 10, 16) = 80$, which divides $560$. Prove that Carmichael numbers must be both "square-free" (not divisible by the square of any prime) and the product of at least three primes. (For this reason, they are not very common.)
        
\end{justbox}

\subsubsection{ENG}

\begin{enumerate}
    \item  Prove that $\lambda(n) \mid \phi(n)$.
    
    $n = p_1^{e_1} \cdots p_r^{e_r}$

    $ \phi(n) = \phi(p_1^{e_1})* \ldots*\phi(p_r^{e_r})$

    $\text{lcm}(\phi(p_1^{e_1}, \ldots, \phi(p_r^{e_r})) | (\phi(p_1^{e_1})* \ldots*\phi(p_r^{e_r}))$
    
    $\lambda(n) \mid \phi(n)$

    \item  Prove that Carmichael numbers must be both “square-free” (not divisible by the square of any prime) 

    let Carmichael number $n = p^\alpha m( \alpha \ge 2 ,  p \nmid m )$
    $a^{n-1} \equiv 1 \pmod{n} (\gcd(a,n) = 1)$
    
    set $a = p+1 $ then $(p+1)^{n} \equiv p+1 \pmod{n}$
    
    and $gcd(p^2,a) = 1$
    
    $(p+1)^{n} \equiv (p+1)^{p^2 p^{\alpha-2}} \equiv p+1 \pmod{p^2}$
    
    but $\gcd(p^2,a) = 1$,  $ a \equiv 1 \pmod{p^2}$
    
    $p+1 \equiv 1 \pmod{p^2}$ This is impossible
    
    \url{https://math.stackexchange.com/questions/1764812/carmichael-number-square-free}


    
    \item  the product of at least three primes. 
    
    Assume that $n=pq$, with $p<q$ two distinct primes, is a Carmichael number. 
    Then we have 
    $q≡1 \pmod{q−1} )\rightarrow n \equiv pq \equiv p \pmod{q−1}  \rightarrow n−1 \equiv p−1 \pmod{q−1}$
    Here $0 < p−1 < q−1$ , so $n−1$ is not divisible by $q−1$.

    
\url{https://math.stackexchange.com/questions/432162/carmichael-proof-of-at-least-3-factors}

\end{enumerate}



\subsubsection{KR}

2. Prove that Carmichael numbers must be both “square-free” (not divisible by the square of any prime) 

\begin{proof}
카마이클 수 $n = p^\alpha m( \alpha \ge 2 ,  p \nmid m )$라 하자
정의에 따라 다음이 성립한다. 
$a^{n} \equiv a \pmod{n}$.

$ a \equiv 1 + p \pmod{p^\alpha}$ 라 하자.

$(p+1)^{n} \equiv p+1 \pmod{m}$이 되는데. 
$m$의 인자인 $p^2$와에 대해서도 다음이 성립한다.
$(p+1)^{n} \equiv (p+1)^{p^\alpha m} \equiv (p+1)^{p^2 p^{\alpha-2} m}  \equiv p+1 \pmod{p^2}$
그러나 $\gcd(a,n) = 1$ 이라서 $\gcd(a, p^2) = 1$이다.
$(p+1)^{n} \equiv p+1  \equiv 1 \pmod{p^2}$이며 이는 모순이다.
\end{proof}



3. the product of at least three primes. 

\begin{proof}
    서로다른 두 소수의 곱이 카마이클 수가 될 수 없음을 보이는것으로 충분하다.

    $n=pq$라 하자( $p<q$인 서로다른 소수)

    $q \equiv 1 \pmod{q - 1} )\rightarrow n \equiv pq \equiv p \pmod{q - 1}$
    $\rightarrow n - 1 \equiv p - 1 \pmod{q - 1}$

    $0 < p-1 < q - 1$ , $n - 1$는 $q - 1$로 나눠질수 없다. 
\end{proof}


%---------------------------------------------------------------------------------------------------------------------------
